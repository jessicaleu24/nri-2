\renewcommand{\thepage}{\arabic{page}}
\invisiblesection{Postdoc Mentoring Plan}
\begin{center}
\vspace{15pt}
\Large{\textbf{Postdoctoral Researcher Mentoring Plan}}
\end{center}

%Each proposal that requests funding to support postdoctoral researchers must include, as a supplementary document, a description of the mentoring activities that will be provided for
%such individuals. In no more than one page, the mentoring plan must describe the
%mentoring that will be provided to all postdoctoral researchers supported by
%the project, irrespective of whether they reside at the submitting organization,
%any subawardee organization, or at any organization participating in a
%simultaneously submitted collaborative project. Proposers are advised that the
%mentoring plan may not be used to circumvent the 15-page project description
%limitation.

\smallskip
%Examples of mentoring activities include, but are not limited to: career
%counseling; training in preparation of grant proposals, publications and presentations;
%guidance on ways to improve teaching and mentoring skills; guidance on how to
%effectively collaborate with researchers from diverse backgrounds and disciplinary areas;
%and training in responsible professional practices. The proposed mentoring activities
%will be evaluated as part of the merit review process under the broader impacts merit
%review criterion. Proposals that include funding to support postdoctoral researchers
%and do not include the requisite mentoring plan will be returned without review.


%Ms. Changliu Liu is currently at the final stage of her PhD study, and I plan to hire her as a post doctoral researcher for this project.  Ms. Liu is determined to develop an academic career, and she has a great potential to become an outstanding educator and researcher.  The primary emphasis during her PhD study was on her research.  Thus, I would like to provide her various opportunities to train her as an effective educator and researcher.  More specifically, my plans are as follows.
PI Tomizuka plans to hire a post doctoral researcher for this project. PI Tomizuka would like to provide the post doctoral researcher various opportunities to train him or her as an effective educator and researcher. More specifically, the plans are as follows.
\begin{enumerate}
\item PI Tomizuka will provide the post doctoral researcher opportunities to supervise graduate students and undergraduate student researchers on the topic of this NSF project. The project has become popular among students and PI Tomizuka is constantly approached by MS and undergraduate students volunteering their time and effort for experience. This does not mean that PI Tomizuka totally delegates to the post doctoral researcher his duty to supervise research students. The post doctoral researcher supervises these students on daily basis, and PI Tomizuka meets the post doctoral researcher and other students at regularly scheduled weekly research meetings and one-on-one individual meetings to provide his advice. 
\item In addition to students involved in this project, PI Tomizuka will let the post doctoral researcher develop interactions with students on other projects and to help their research. This will give the post doctoral researcher opportunities to improve skills for interacting with student, to learn other research projects and to develop a broad perspective on mechatronics research.  Other projects actively pursued in PI Tomizuka's research group include: autonomous driving, intelligent control of robot manipulators, building temperature control, and mechatronics for human assistance.
\item In April every year, University of California at Berkeley hosts CalDay, an open house of the Berkeley campus to general public including K-12 students.  PI Tomizuka's laboratory is a participant of this event every year. PI Tomizuka will ask the post doctoral researcher to play a leading role in this activity. By serving as the leader of PI Tomizuka's research group, the post doctoral researcher will be able to develop deep understanding of outreach activities.  
\end{enumerate}


%\bigskip
%\textbf{Documentation of collaborative arrangements} of significance to the
%proposal through letters of commitment.
%
%\smallskip
%%Proposers are reminded that, unless required by a specific program solicitation,
%%letters of support should not be submitted as they are not a standard component
%%of an NSF proposal, and, if included, a reviewer is under no obligation to review
%%these materials. Letters of support submitted in response to a program solicitation
%%requirement must be unique to the specific proposal submitted and cannot be altered
%%without the author's explicit prior approval. NSF may return without review proposals
%%that are not consistent with these instructions.
%
%\bigskip
%\textbf{Documentation regarding research involving the use of human subjects}, hazardous
%materials, vertebrate animals, or endangered species.
