\setcounter{page}{1}
\renewcommand{\thepage}{\arabic{page}}
\invisiblesection{Human Subjects Protection Plan}

~
\vspace{-15pt}
\begin{center}
\Large{\textbf{Human Subjects Protection Plan}}
\end{center}
\subsection{Risks to Subjects}

We will include human subject participation in three ways: (1) construction of the cognition model library in T1, (2) construction of the motion skill library in T2, (3) evaluation of the SERoCS on Platforms 1-3 in T4, and (4) evaluation of the SERoCS on Platform 5 in T4.

\textbf{Construction of the cognition model library in T1}. The experiment in collecting human motion data will be performed inside PI Tomizuka's Mechanical Systems Control Lab at UC Berkeley. The subject's motion will be captured by vision cameras. There is no risk to the subject during the study.

\textbf{Construction of the motion skill library in T2}. The human demonstration will be performed inside PI Tomizuka's Mechanical Systems Control Lab at UC Berkeley. The subject will be asked to (1) wear motion capture suits or gloves, and perform remote demonstrations of multiple types of tasks, (2) grasp the robot end-effector, and perform direct demonstrations by lead through teaching. There is a potential risk of physical injury from collision between the robot and the subject. Several safeguarding mechanisms, as will be described below, will be in place to reduce such risks.

\textbf{Evaluation of the SERoCS on Platforms 1-3 in T4}. The evaluation and testing will be performed inside PI Tomizuka's Mechanical Systems Control Lab at UC Berkeley. The subject interacts with the robot virtually. There is no risk to the subject during the study.

\textbf{Evaluation of the SERoCS on Platform 5 in T4}. The evaluation and testing will be performed inside PI Tomizuka's Mechanical Systems Control Lab at UC Berkeley. The subject will be asked to perform (1) simple tasks (sitting on a chair, walking around) inside the robot's workspace and (2) joint tasks (lift/move/rotate one workpiece) with the robot. There is a potential risk of physical injury from collision between the robot and the subject. Several safeguarding mechanisms, as will be described below, will be in place to reduce such risks.

\textbf{Confidentiality}. There is a potential risk of loss of confidentiality from participation in these experiments.

\subsection{Plans for Recruitment and Informed Consent}
Subjects will be evaluated in PI Tomizuka's Mechanical Systems Control Lab at UC Berkeley. 5-10 healthy adult volunteers (ages 18-60) will be recruited for the experiment, including 3-5 experienced workers who are familiar with robotics.

All subjects will undergo informed consent procedures based on the approval of the University Institutional Review Board (IRB). Consent documents will present a complete review of study background, aims, procedures, potential risks and benefits, study costs, rights of research participants, and study staff contact information. The experiments will be conducted in accordance with regulatory requirements regarding involvement of human subjects in research. All subject testing protocols, consent documents, and patient materials will be reviewed and approved by the UC Berkeley Committee for Protection of Human Subjects.

\subsection{Inclusion of Women, Minorities, and Children}
The inclusion of women, minorities, and children will be considered as follows:

\textbf{Inclusion of Women}. Both men and women will be recruited for this study, whereas equal number of
men and women will be included, for an adequate sample size of men and women to address gender differences with respect to evaluation procedures and task performance.

\textbf{Inclusion of Minorities}. The study does not exclude volunteers on the basis of race or ethnicity. The demographic mix or racial and ethnic groups is well represented at UC Berkeley as well as in the local communities. Attempts will be made to recruit subjects with a similar ethnic and racial diversity to that of the general population in the respective areas.

\textbf{Inclusion of Children}. Since the project is aimed at evaluating robots' safety performances in a factory environment, we do not plan to include individuals of less than 18 years of age.


\subsection{Planned Procedures to Protect Against and Minimize Potential Risks}
To protect against and to minimize potential risks to human subjects, the following mechanisms will be in place:

\textbf{Construction of the motion skill library}.
(1) Emergency check (including threshold monitoring of joint position, velocity and torque, collision prediction and detection)  and various software failure protocols will be developed and implemented under different task scenarios. The robot should stop immediately if emergency or software failure is detected.  (2) Prior to the experiments with the robot, the human subjects will be trained (i) using the simulation platforms, e.g. Platform 1 and Platform 2, and (ii) by indirect interaction with the robot arm in Platform 3, i.e. requiring the subjects to control a mobile robot to interact with the robot arm. (3) Subjects will be required to wear helmets and goggles during the experiments. (4) A research staff will monitor the experiment and will stop the robot in emergencies. (6) Questionnaires will be sent out after the tests to address potential issues related to subject's comfort and acceptance of the collaborative robot and the SERoCS.

\textbf{Evaluation of the SERoCS on Platform 5 with the mobile robot}. (1) The experiment will be performed only if the SERoCS demonstrates success in the Platforms 1-3. (2) Prior to the experiments with the mobile robot, the human subjects will be trained (i) using the simulation platforms, e.g. Platform 1 and Platform 2 and (ii) by indirect interaction with the testing mobile robot in Platform 3, i.e. requiring the subjects to control another mobile robot to interact with the testing mobile robot. (3) Subjects will be required to wear helmets and goggles during the experiments. (4) Emergency checks and various software failure protocols will be implemented under different task scenarios. (5) A research staff will monitor the experiment and will take charge of the control of the testing mobile robot in emergencies. (6) Questionnaires will be sent out after the tests to address potential issues related to subject's comfort and acceptance of the SERoCS.

\textbf{Evaluation of the SERoCS on Platform 5 with the robot arm}. 
(1) The experiment will be performed only if the SERoCS demonstrates success in the simulation platforms. (2) Prior to the experiments with the robot arm, the human subjects will be trained (i) using the simulation platforms, e.g. Platform 1 and Platform 2, and (ii) by indirect interaction with the robot arm in Platform 3, i.e. requiring the subjects to control a mobile robot to interact with the robot arm. (3) Subjects will be required to wear helmets and goggles during the experiments. (4) A research staff will monitor the experiment and will stop the robot in emergencies.
(5) Emergency check and various software failure protocols will also be implemented under different task scenarios. (6) Questionnaires will be sent out after the tests to address potential issues related to subject's comfort and acceptance of the collaborative robot and the SERoCS.

\textbf{Confidentiality}. No personal identifiable information will be collected on the subjects. In addition to the sensor data we will collect anthropometric data (e.g., height), demographic data (e.g., age, sex) and data indicating whether the subject is experienced in the field of robotics or not. All the data will be anonymized in the process. The data will be stored on secured servers. Only the PI Prof. Tomizuka, and the researcher(s) working on this project will have access to the personal data. This information will be stored in locked cabinets and/or password protected computers in their respective offices. For the public dissemination of the sensor data, we will minimize potential risks of loss of confidentiality by coding all subjects and their recordings.





